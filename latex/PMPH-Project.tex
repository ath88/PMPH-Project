\documentclass[11pt]{article}

%---- defitions ----
\def\Title{Programming Massively Parallel Hardware\\
\vspace{1.5cm}
\textbf{Group Project}}
\def\Author{Esben Skaarup, Asbj\o rn Thegler \& \'{A}sbj\o rn Vider\o \ J\o kladal}

%---- packages ----
\usepackage[english]{babel}
\usepackage[utf8]{inputenc}
\usepackage{courier}
\usepackage{listings}
\usepackage[pdftex,colorlinks=true]{hyperref}

\begin{document}
\title{\Title}
\author{\Author}
\date{\today}
\maketitle

We have chosen to first reason about the original sequential implementation. 
Next, we reason about and explain how we solved the OpenMP 
implementation. Finally, we reason about our CUDA implementation, and measure
it against the original implementation and the OpenMP implementation.

\section{Sequential Implementation}
\subsection{Timing}
We have made approximate timing of specific parts of the original 
implementation, to get an overview 
of where we can achieve relevant speedups according to Amdahl's Law. The results
can be seen in \autoref{table:origtime}. In the table, there are indentations 
in the first column indicating what parts are inside other parts. For example,
the running time of tridag\_0 is part of the running time of rollback\_2, which
in time is part of rollback, and so forth. The time of the outer loop is the 
total running time of the entire calculation. 

\begin{table}[h]
\centering
\begin{tabular}{|l|l|l|l|l|}
\hline
Name \textbackslash\ Dataset & Small      & Medium     & Large        & Avg. Percentage \\ \hline
outer                        & 2050766 us & 4240619 us & 187729378 us & 100\%           \\ \hline
\ \ init                     & 171 us     & 238 us     & 2068 us      & \\ \hline
\ \ setPayoff                & 174 us     & 363 us     & 10651 us     & \\ \hline
\ \ updateParams             & 1199969 us & 2492123 us & 106691216 us & \\ \hline
\ \ rollback                 & 839027 us  & 1729438 us & 80412624 us  & \\ \hline
\ \ \ \ rollback\_0          & 117198 us  & 246938 us  & 11824914 us  & \\ \hline
\ \ \ \ rollback\_1          & 111773 us  & 236186 us  & 12284147 us  & \\ \hline
\ \ \ \ rollback\_2          & 286646 us  & 585957 us  & 25915774 us  & \\ \hline
\ \ \ \ \ \ tridag\_0        & 203364 us  & 421277 us  & 18500994 us  & \\ \hline
\ \ \ \ rollback\_3          & 304214 us  & 627087 us  & 28554403 us  & \\ \hline
\ \ \ \ \ \ tridag\_1        & 200502 us  & 416905 us  & 18612169 us  & \\ \hline
\end{tabular}
\caption{Approximate timings of the original implementation}
\label{table:origtime}
\end{table}

\subsection{Validation}
The original implementation validates against all 3 datasets. This is expected,
since we did not change the original implementation. It is assumed that the 
original implementation is therefore correct.


\section{OpenMP Implementation}
\subsection{Privatization}
The outer loop in the original implementation uses the same allocation of a C 
struct for each
iteration of the loop. This, in turn, makes it inherently not parallel, since,
if executed in parallel, all iterations would be writing to the same memory. 
We can, however, privatize the struct, such that all iterations allocate their
own struct, and therefore does not read and write to the same memory. This is 
a method known as privatization, and allows us to parallelize the entire loop
with an OpenMP pragma directive.

\subsection{Validation}
The implementation validates against all 3 datasets. This shows that our
implementation is not catastrophically wrong, but does not prove that it is
correct. We can make the same assumption about correctness as with the 
original implementation.

\subsection{Speedup}
While this implementation is parallel across multiple processors, measuring the
time taking of specific parts of the implementation is harder. We can, however, 
compare with the total running time, and the results can be seen in TODO FIGURE.



\section{CUDA Implementation}

\subsection{Transformations}
\subsection{Validation}
\subsection{Speedup}

\newpage
\section{Appendices}
\subsection{Example output}
Here follows example output from running the different implementations.

\subsection{Original implementation, small dataset}
\lstinputlisting[basicstyle=\footnotesize\ttfamily,breaklines=true]{data/orig-run-small.txt}
\subsection{OpenMP implementation, small dataset}
\lstinputlisting[basicstyle=\footnotesize\ttfamily,breaklines=true]{data/OpenMP-run-small.txt}
\subsection{Original implementation, medium dataset}

\lstinputlisting[basicstyle=\footnotesize\ttfamily,breaklines=true]{data/orig-run-medium.txt}
\subsection{OpenMP implementation, medium dataset}
\lstinputlisting[basicstyle=\footnotesize\ttfamily,breaklines=true]{data/OpenMP-run-medium.txt}
\subsection{Original implementation, large dataset}

\lstinputlisting[basicstyle=\footnotesize\ttfamily,breaklines=true]{data/orig-run-large.txt}
\subsection{OpenMP implementation, large dataset}
\lstinputlisting[basicstyle=\footnotesize\ttfamily,breaklines=true]{data/OpenMP-run-large.txt}

\end{document}
